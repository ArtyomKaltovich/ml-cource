Судя по тому, что вы решили почитать эту книгу, вам приходилось слышать о машинном обучении, пожалуй, сейчас сложнее не услышать это словосочетание, например, термин "Machine Learning"\ содержится в \href{https://en.wikipedia.org/wiki/List_of_buzzwords}{списке buzzword'ов английской википедии}.
Модные слова (также гламурная лексика и «умные слова», англ. buzzword)~— особый род новых слов и речевых конструкций, часто используемых в коммерции, пропаганде и профессиональной деятельности для оказания впечатления осведомлённости говорящего и для придания чему-либо образа важности, уникальности или новизны. Из-за неумеренного употребления смысл слова размывается, и «модные слова» можно встретить даже в контексте, не имеющем отношения к исходному значению: например, «элитные семинары [слово, в какой-то момент заменившее «элитарные» — доступные самым богатым] по умеренным ценам», «эксклюзивные часы [изготовленные штучно], выпущенные тиражом в 11111 экземпляров». \cite{wiki:buzzword_def}\\

Машинное обучение попало в этот список не просто так, действительно часто в СМИ появляются новости, что ещё один рубеж покорился программе, иногда можно услышать апокалиптические сценарии о скором восстании машин или наоборот, идеалистические, о скором приходе технологической сингулярности и возможности перенесения сознания в машину или воссоздания сознания умерших людей. Мы же будем в стороне от всего этого ажиотажа и подойдём непосредственно к изучению основ машинного обучения. Что это? Зачем нужно? Как и где используется? Главная же цель \-- помочь вам изучить основы машинных алгоритмов.\\

